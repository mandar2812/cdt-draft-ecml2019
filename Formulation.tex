\section{Position of the problem}\label{sec:formulation}

Time lag inference is essentially a regression problem with two tasks. 
Given two time series, the causes $x(t)$ and the observed effects $y(t)$, the regression model 
must learn a mapping $f(.)$ which maps each input pattern $x(t_1)$ to an output $y(t_2)$, and a 
mapping $g(.)$ which maps the time delay between the input and output patterns $t_2 = t_1 + g(x(t_1))$. 
This is formally specified in equations below.

\begin{align}
y(t + \tau(t)) & = f[x(t)]\label{eq:pb1}\\
\tau(t) & = g[x(t)]\label{eq:pb2} 
\end{align}
with
\[
f: \mathcal{X}  \rightarrow \mathbb{R},\qquad\text{and}\qquad
g: \mathcal{X}  \rightarrow \mathbb{R}^{+},
\]
$t \in \mathbb{R}^{+}$ represents the continuous temporal domain. The input signal 
$x(t)\in \mathcal{X}$ is possibly high dimensional and contains the hidden cause to 
the effect $y(t)\in\mathbb{R}$ which is considered as a scalar in this paper. 
$\tau(t)\in \mathbb{R}^+$ represents the time delay between inputs and outputs.

It is important to note some caveats about this time lag which make this inference problem 
different from classical time series regression problems.

Firstly, the time lag $\tau(t)$ can be \emph{non-stationary} i.e. dependent on the input $x(t)$. 
Secondly, in practical applications such as the motivating space weather example, $\tau(t)$ 
is not recorded explicitly in the training data. This makes model training and evaluation 
challenging.

One may make certain assumptions on the time warping function $\phi(t) = t + \tau(t)$, in the general 
formulation of the problem presented here: 

\begin{enumerate}
    \item $\phi(.)$ is assumed to be continuous and \emph{bijective}.
    \item Zhou \& Sornette \cite{ZHOU2006195} enforce further assumptions 
          such as $\phi(t_1) \leq \phi(t_2), \forall t_1 \leq t_2$, which 
          are incorporated via constraints. This approach respects the 
          properties associated with the warping function $\phi(.)$ but 
          increase the computational complexity of the resulting 
          optimization problem.
\end{enumerate}

The second assumption can be restrictive for applications in space physics, where fast moving 
solar wind can reach the Earth before slow moving solar wind which departed before it. 

In practical time-series applications, one works with sub-sampled or discretized versions of the time 
series $x(t)$ and $y(t)$. In this case the map $g(.)$ is replaced by its discrete analogue 
$g(.):\mathcal{X}  \rightarrow \mathbb{N}$. 

Henceforth we will work with the discretized time series $x^m$ and $y^m$.


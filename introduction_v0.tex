\section{Introduction}
Exploiting causal relationship between time series is a problem of practical importance 
with many domains of application. Of similar importance is causal relationship between processes. 
Although rigorous understanding and identification of causal links between arbitrary quantities 
is an elusive question, for certain situations it is well known that some causal relationship 
must exist. 

Some of the seminal work relating to causality in time series data is found in \cite{Granger}, 
which introduced the concept of \emph{Granger Causality}. \emph{Granger causality} focused on the 
problem of predictive causality instead of the philosophical or \emph{true causality}. Although 
originally formulated for linear causal relationships with no latent variables, it has since been 
extended for various situations (see \cite{doi:10.1002/9781119945710.ch22} for a review).

In temporal phenomena, it is also the case that causal effects of events are not immediately observed, 
but after a certain time interval which can be dynamic. One prominent example of such behavior is the
 Sun-Earth system and its associated problem of space weather forecasting.

The Sun, a perennial source of charged energetic particles ejects them into the surrounding space. 
This particle cloud or \emph{solar wind} reaches the Earth's vicinity and interacts with its 
magnetic field in complex ways, giving rise to geomagnetic phenomena. High speed solar wind can 
potentially cause damage to under sea pipelines, satellites, and other telecommunication 
infrastructure. A key prediction task is to forecast the speed of the \emph{solar wind} in 
the vicinity of the Earth from solar image data 
(\cite{doi:10.1002/jgra.50429}, \cite{doi:10.1029/2009SW000542}).

The \emph{solar wind} forecasting problem can be broken down into two stages. First, the extraction of 
features from solar images and second, the prediction of time lagged solar wind speed near the Earth. 
In this work we present progress made on the second part of this problem.
 
To our knowledge, the problem of regression with hidden non-constant time delay, as formulated later in 
section \ref{sec:formulation} has not been addressed in the machine learning literature. 

Similar type of problems have actually been encountered in the context of financial time series 
prediction in \cite{ZHOU2006195} for instance. Their approach is a form a dynamical time warping (DTW) 
which appeared originally in the context of speech recognition~\cite{SakoeShiba1978}. 
\cite{SignalDiffusion} build on the DTW paradigm and take a Bayesian approach to the temporal alignment 
problem between time series. However they limit themselves to linear relationships between 
the input and output time series, assuming in addition slow varying time lags.

The DTW algorithm and its variants are now widely used in time series analysis, but they always assume 
a predefined cost matrix for the temporal alignment between two time series and they assume the 
causes $x(t)$ and effects $y(t)$ are of same dimensionality and structure. 

Our work has the following novel characteristics as compared to the DTW paradigm.

\begin{enumerate}
    \item Causes and effects are of different dimensionality. The output time series $y(t)$ is scalar, 
          but its driver $x(t)$ is potentially high dimensional, e.g. vectors, images etc.
    \item We do not make assumptions as made in \cite{ZHOU2006195} and other 
          \emph{dynamic time warping} related works about the time lag function. 
          In our case the time lag can be potentially non smooth.
    \item The relationship between $x(t)$ and $y(t)$ can be potentially non-linear. 
\end{enumerate}

